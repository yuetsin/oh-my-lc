%-*- coding: UTF-8 -*-

\documentclass[UTF8]{ctexart}
\usepackage{graphicx}
\usepackage{float}
\usepackage{CJKpunct}
\usepackage{amsmath}
\usepackage{geometry}
\geometry{a6paper,centering,scale=0.8}
\usepackage[format=hang,font=small,textfont=it]{caption}
\usepackage[nottoc]{tocbibind}
\setromanfont{STSongti-SC-Regular} %设置中文字体
\punctstyle{quanjiao} %使用全角标点	


\title{Problem 343}
\author{Integer Break}


\begin{document}

把一个 2 ~ 58 之间的整数拆成若干个正整数的和,让他们的乘积最大。

首先限定 58 也是为了照顾某些 Int 限高的语言…因为 $integerBreak(59)$ 就超过了 $2147483647$ 了。

然后因为输入过小…可以直接打表了。

Runtime: 24 ms, faster than 100.00\% of Python3 online submissions for Integer Break.

Memory Usage: 13.8 MB, less than 11.11\% of Python3 online submissions for Integer Break.

事实上,有人类的解决方案:

首先,因为加和乘操作都是可交换的。也就是说我们的 IntegerSplit 和结果都是可分拆的。

因此我们只需要把一个整数拆成两部分(如果可以?),然后递归调用自己拿到两个孩子的最大乘积,再把他们乘起来就好了!Perfect!

不过这样会多次重复调用,因此我们最好还是使用 DP 法来解决。

以上。

我爱 DP,没人爱我。

\end{document}